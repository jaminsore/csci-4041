\documentclass[11pt]{article}  % 11 points is a good size
\usepackage{fullpage}

\begin{document}
\title{Week 4 In-Class}
\author{
    Benjamin Sorenson \\
    \and
    Ailin Deng \\
    \and
    Katherine Yang
}

\maketitle
\section{Question 2}
The liklihood of an element being selected as the pivot point is part of the
analysis because how the array is partitioned affects the running time. If
there is at least one element on the other side of the pivot, the running
time is \(\Theta\left(n\lg n\right)\), but if one sub-array is empty, the
running time is \(\Theta\left(n^2\right)\). So, in the analysis we need to
consider the liklihood that \(p\) is either the largest or smallest element
of \(A\). In an array of size \(n\), the liklihood of selecting element
\(i\) is \(\frac{1}{n}\).

\section{Question 4}
If we left \(X_i\) be the event that person \(i\) received the correct hat, then
\[ Pr\left\{X_i=1\right\} = \frac{1}{n}\]
\[E\left(X\right)=\sum_{i=1}^{n}{\frac{1}{n}}=n\times\frac{1}{n}=1\]
\end{document}