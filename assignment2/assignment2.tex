\documentclass[11pt]{article}
\usepackage{fullpage}
\usepackage{algorithm}
\usepackage{algorithmicx}
\usepackage{algpseudocode}
\begin{document}
\author{Benjamin Sorenson} \title{Assignment 2}
\maketitle


\section{2.3-7}

We could use \texttt{MERGE-SORT} to sort \(S\)
in \(\Theta(n\lg n)\)
time, and then for each \(s \in S\)
use \texttt{BINARY-SEARCH} to search for \(s-x\).
\texttt{BINARY-SEARCH} is defined below, and on each call it divides
the problem by two and makes a recursive call until, in the worst
case, the length of the array is \( \le 2\)---so
it's worst-case runtime is \( O( \lg n)\).
The function \texttt{FIND-EXACT-SUM} implements the the described
algorithm (I haven't quite got the hang of \LaTeX formatting so the algorithms appear on the next page).

\begin{algorithm}
  \begin{algorithmic}
    \Function{binary-search}{$A, l, r, k$} \State
    $mid \gets \lfloor \frac{l + r}{2} \rfloor$
    \If{$A\left[mid\right] = k$} \State \Return $mid$ \EndIf
    \If{$mid=l$}\If{$A\left[mid\right] > k$}\State \Return $mid$ \Else
    \State \Return $mid + 1$\EndIf\EndIf \If{$A\left[mid\right] < k$}
    \State \Return \Call{binary-search}{$A,l,mid,k$} \EndIf
    \If{$A\left[mid\right] > k$} \State \Return
    \Call{binary-search}{$A,mid+1,l,k$} \EndIf
    \EndFunction
  \end{algorithmic}
  \begin{algorithmic}
    \Function{find-exact-sum}{$S, x$}\State \Call{merge-sort}{$S$}
    \For{$i \gets 1$ to $length \left[S\right]$} \State
    $d \gets S\left[i\right]- x$ \State
    $j \gets $\Call{binary-search}{$S, 1, length\left[S\right], d$}
    \If{$S\left[j\right] = x$} \State \Return \texttt{True}\EndIf
    \EndFor
    \State \Return \texttt{False}
    \EndFunction
  \end{algorithmic}
\end{algorithm}
\section{6.3-3}
The number of leaves in a tree of size \(n\)
is \(\left\lceil \frac{n}{2}\right\rceil\).
Without loss of generality, we will assume that \(n = 2^k\).
The number of nodes one level up from the bottom (\(h=1\)),
can be found by subtracting the number of leaves from the total number
of elements and applying the same formula. This sets up the recurrence
relation
\[\left\lceil\frac{ x_n}{2}\right\rceil = \left\lceil\frac{x_{n-1} - \left\lceil \frac{x_{n-1}}{2}
    \right\rceil}{2} \right\rceil\]
with the initial condition \(x_0 = 2^k\), this becomes
\[x_n = x_{n-1} - \frac{x_{n-1}}{2}\]
Assume that for some \(x_r\), \[x_r \le \frac{2^k}{2^{r}}\] Then
\[x_{r+1} = x_r - \frac{x_r}{2} \le
\frac{2^k}{2^{r}} - \frac{\frac{2^k}{2^{r}}}{2}\]
\[x_{r+1} \le \frac{2^k}{2^{(r+1)}} \]
\section{6.5-8}
The idea begin \texttt{HEAP-DELETE} is that we can use two
\(\Theta (\lg n)\)
procedures, \texttt{HEAP-INCREASE-KEY} and \texttt{HEAP-EXTRACT-MAX}
to delete an element of the heap \(A\).
First, increase the key to \(\max + 1\),
and then use \texttt{HEAP-EXTRACT-MAX} to remove it.

\begin{algorithm}
  \begin{algorithmic}
    \Procedure{heap-delete}{$A,i$} \State $max \gets$
    \Call{heap-max}{$A$}\State \Call{heap-increase-key}{$A, i, max+1$}
    \State \Call{heap-extract-max}{$A$}
    \EndProcedure
  \end{algorithmic}
\end{algorithm}
\section{6.5-9}
I couldn't quite figure this one out.  I know that we could perform
\texttt{HEAP-EXTRACT-MIN} in \(\Theta(\lg n)\)
time on each of the \(k\)
sub-lists, but this wouldn't guarantee that we're getting the lowest
values from all \(k\)
sub-lists.  The best I could come up with was to find the \(\min\)
of the \(\min\)s
from each of the \(k\)
sub-lists, extract it from its sub-list, and place it in the merged
array, but that would be \(\Theta(nk\lg k)\)
\section{7.2-1}
\begin{description}
\item[Upper Bound] Show \(T(n) \le c n^2\)
  \begin{description}
  \item[Assume] \(T(n-1) \le c (n-1)^2\)
  \item[Then] \(T(n) \le c n^2 - 2 cn + c + c_1 n \le cn^2\)
    for \(c_1 < 2c\) and \(n \ge 1\)
  \end{description}
\item[Lower Bound] show \(T(n) \ge c n^2\)
  \begin{description}
  \item[Assume] \(T(n-1) \ge c(n-1)^2\)
  \item[Then] \(T(n) \ge cn^2 -2cn + c + c_1n \ge cn^2\)
    for \(c_1 \ge 2c\) and \(n \ge 1\)
  \end{description}
\item[Base Case] \(n=1\), \(T(1) = c = c (1^2)\)
\end{description}
\section{7.3-1}
We analyze the expected running time of a randomized algorithm because
as \(n\)
gets large the the probability of the worst-case running
time approaches \(0\)
so the expected running gives a more realistic asymptotic behavior.
\end{document}