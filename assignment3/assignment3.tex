\documentclass[11pt]{article}
%\usepackage{fullpage}
\usepackage{enumitem}
\usepackage{pdfpages}
\usepackage{graphicx}
\usepackage{amsthm}
\usepackage{amsmath}
\usepackage{algorithm}
\usepackage{algorithmicx}
\usepackage{algpseudocode}
\begin{document}

\author{Benjamin Sorenson} \title{Assignment 3}
\maketitle

\begin{enumerate}[label=\bfseries Question \arabic*:]
\item There is no comparison sort whose running time is linear for at
  least half of the \(n!\)
  inputs. From the proof of theorem 8.1,
  \(\lg\left(\frac{n!}{2}\right)\)
  is still \(\Omega(n \lg n)\).
  Similarly, \(\lg\left(\left(n-1\right)!\right)\)
  is also still \(\Omega(n \lg n)\).
\item
  \begin{enumerate}
  \item By using this algorithm we can gaurentee that we avoid the
    worst case for \texttt{QUICKSORT}.  We can add a call to
    \texttt{SELECT} to select the pivot corresponding to the
    \(i^{th}\)
    order statistic (the median for example) for the input array
    \(A\),
    avoiding the worst-case scenario for \texttt{QUICKSORT}---the case
    when there are \(n-1\)
    elements on one side of the pivot and \(0\)
    on the other.  Since \texttt{PARTITION} and \texttt{SELECT} run in
    \(\Theta(n)\)
    time this means that we never have to worry about the worst case
    (\(\O(n^2)\)),
    and we don't increase in the asymptotic running time of
    \texttt{PARTITION}.
  \item This bound is preferred because it is a guarantee not a
    probabilistic expectation.
  \end{enumerate}
\item
  \begin{enumerate}
  \item
    \[ m\left[i,j\right] = \left\{\begin{array}{ll}0 & \mbox{if
          $i = j$}; \\ \max\limits_{i \leq k < j}{\left\{m[i,k] +
            m[k+1, j] + p_{i-1}p_kp_j\right\}} & \mbox{if
          $i<j$}. \\ \end{array}\right. \]
  \item
    \begin{proof} For the sake of contradiction, assume that there is
      some parenthesization of the first \(k\)
      (\(A_i \dots A_k\))
      and the last \(n-k\)
      (\(A_{k+1} \dots A_n\))
      arrays such that
      \(m\left[1,n\right] = m\left[1,k\right] + m\left[k+1, n\right] +
      p_0p_kp_n \)
      is the maximum number of possible scalar operations, and there
      exists some \(m[1, k]' > m[1, k]\)
      then
      \(m\left[1,n\right] < m\left[1,k\right]' + m\left[k+1, n\right]
      + p_0p_kp_n \)
      which contradicts the fact that \(m[1, n]\)
      is the maximum number of scalar operations.  By the same
      reasoning, a contradiction also arises if there exists an
      \(m[k+1, n]' > m[k+1, n]\).\end{proof}
  \end{enumerate}
\item
  \begin{algorithm}
    \begin{algorithmic}
      \Function{memoized-lcs-length}{$X, Y$} \State
      $m \gets length\left[X\right]$ \State
      $n \gets length\left[Y\right]$

      \For{$i \gets 0$ \textbf{to} $m$} \For{$j \gets 0$ \textbf{to}
        $n$} \State $c\left[i, j\right] \gets -\infty$
      \EndFor
      \EndFor
      \State \Return \Call{lookup-lcs-length}{$X, Y, n, m$}
      \EndFunction
    \end{algorithmic}

\begin{algorithmic}
  \Function{lookup-lcs-length}{$X, Y, i, j$}
  \If{$c\left[i,j\right] >- \infty$} \State\Return $c\left[i,j\right]$
  \EndIf
  \If{$i=0 \lor j=0$} \State$c\left[i,j\right] \gets 0$
  \ElsIf{$X\left[i\right]=Y\left[j\right]$} \State
  $c\left[i,j\right] \gets$
  \Call{lookup-lcs-length}{$X, Y, i-1, j-1$}$+1$ \Else \State
  $left \gets$ \Call{lookup-lcs-length}{$X, Y, i, j-1$} \State
  $up \gets$ \Call{lookup-lcs-length}{$X, Y, i-1, j$} \State
  $c\left[i, j\right] \gets \max{\left\{left, up\right\}}$
  \EndIf
  \State \Return $c\left[i,j\right]$
  \EndFunction
\end{algorithmic}

\end{algorithm}

\item
  \begin{algorithm}
    \begin{algorithmic}
      \Procedure{print-lcs}{$c, X, i, j$} \If{$i=0 \lor j=0$} \State
      \Return\EndIf
      \If{$c\left[i-1, j\right] = c\left[i, j-1\right]=c\left[i-1,
          j-1\right]$}
      \State\Call{print-lcs}{$c, X, i-1, j-1$} \State print
      $X\left[i\right]$
      \ElsIf{$c\left[i-1,j\right] \geq c\left[i, j-1\right]$}\State
      \Call{print-lcs}{$c, X, i-1, j$} \Else \State
      \Call{print-lcs}{$c, X, i, j-1$}
      \EndIf
      \EndProcedure
    \end{algorithmic}
  \end{algorithm}
\item
  \begin{enumerate}
  \item A brute force solution would be to iterate over every possible
    combination of exchanges of currencies starting with \(c_1\)
    and ending with \(c_n\)
    of lengths \(1 \dots n-1\).
    Since there are \(n-1\)
    exchange rates in each sequence, the running time of this approach
    is given by the equation \(\sum_{k=1}^{n-1}{\binom{n-1}{k}}\).
    Letting \(m = n-1\),
    this becomes
    \(\sum_{k=1}^{m}{\binom{m}{k}}=\sum_{k=0}^{m}{\binom{m}{k}}
    -1=2^m-1=2^{n-1}-1=\Theta\left( 2^n\right)\).
    This approach doesn't take into account that each exchange must
    happen in order (that is, \(c_i\)
    must be exchanged with \(c_{i+1} \dots c_{n}\)),
    but an approach that did would still be exponential.
  \item
    \[ P\left(i\right)=\left\{\begin{array}{ll} 1 & \mbox{if $i=1$};
        \\ \max\limits_{1 \leq k <
          i}{\left\{P\left(i-k\right)r_{i-k,i}\right\}} & \mbox{if
          $i > 1$}. \\ \end{array} \right. \]
  \item Suppose there was an optimal sequence of \(m\)
    exchanges such \(P(n) = P(k)r_{kn} - \sum_{i=1}^{m}{C_i}\)
    where \(C_i\)
    is the commission charged at exchange \(i\)
    yielded the maximum exchange rate. Consider \(P(k)' > P(k)\).
    We cannot guarantee that substituting \(P(k)'\)
    for \(P(k)\)
    would result in a higher return since we don't know the commission
    charged for \(P(k)'\).
  \end{enumerate}
                        \end{enumerate}
x\end{document}